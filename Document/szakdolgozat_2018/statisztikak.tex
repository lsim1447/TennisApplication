%!TEX root = dolgozat.tex
%%%%%%%%%%%%%%%%%%%%%%%%%%%%%%%%%%%%%%%%%%%%%%%%%%%%%%%%%%%%%%%%%%%%%%%
\chapter{Statisztikák}\label{ch:MAT}

\begin{osszefoglal}
	Az alkalmazás fejlesztése folyamán több alkalommal leteszteltük az alkalmazás különböző funkcióit, lásd: tréning és tesztelés. Ezeket különböző számú és alakú bemeneti adatokkal próbáltuk ki, viszont természetesen az algoritmusok hatékonyságán és működésén is folyamatosan alakítgattunk. Ebben a rövid fejezetben ezeknek a fázisoknak az eredményeit és az ezzekkel kapcsolatos észrevételeinket mutatjuk be.
\end{osszefoglal}


%%%%%%%%%%%%%%%%%%%%%%%%%%%%%%%%%%%%%%%%%%%%%%%%%%%%%%%%%%%%%%%%%%%%%%%
\section{Első mérföldkő/méréspont}\label{sec:MAT:bev}
\paragraph{}
Ebben a fázisban mindössze 18 input adatot használtunk mérkőzésenként:
\begin{enumerate}
\item[•] a legutolsó N mérkőzés győzelmi aránya játékosonként - összes (2 érték)
\item[•] a legutolsó M mérkőzés győzelmi aránya játékosonként - az adott talajon (2 érték)
\item[•] a legutolsó P mérkőzés győzelmi aránya játékosonként - az adott tornán (2 érték)
\item[•] a legutolsó N mérkőzésen a nyert játékok és játszmák győzelmi aránya játékosonként - összes (4 érték)
\item[•] a legutolsó M mérkőzésen a nyert játékok és játszmák győzelmi aránya játékosonként - az adott talajon (4 érték)
\item[•] az egymás elleni R mérkőzés győzelmi aránya játékosonként - összes (2 érték)
\item[•] az egymás elleni S mérkőzés győzelmi aránya játékosonként - az adott talajon (2 érték)
\end{enumerate}

\begin{center}
\begin{tabular}{ |p{2cm}|p{2cm}|p{2cm}|p{2cm}|p{2cm}|  }
 \hline
 \multicolumn{5}{|c|}{A fent használt mennyiségek számokban} \\
 \hline
  N & M & P & R & S\\
 \hline
 160 & 100 & 60 & 12 & 8 \\
 \hline
\end{tabular}
\end{center}

\paragraph{}
A fenti adatokat és szempontokat felhasználva/figyelembe véve a következő eredményeket kaptuk: 
\begin{center}
\begin{tabular}{ |p{4cm}|p{4cm}|  }
 \hline
 \multicolumn{2}{|c|}{A training és a teszt eredménye százalékban - 5000 epoch} \\
 \hline
  Training & Teszt\\
 \hline
 64 & 62  \\
 \hline
\end{tabular}
\end{center}

%%%%%%%%%%%%%%%%%%%%%%%%%%%%%%%%%%%%%%%%%%%%%%%%%%%%%%%%%%%%%%%%%%%%%%%
\section{Második mérföldkő/méréspont}\label{sec:MAT:bev}
\paragraph{}
Ebben a fázisban 34 input adatot használtunk mérkőzésenként. Az előzőhöz képest újonnal hozzáadott szempontok:
\begin{enumerate}
\item[•] a legutolsó N mérkőzés: adogatási arány játékosonként - összes (2 érték)
\item[•] a legutolsó N mérkőzés: fogadó arány játékosonként - összes (2 érték)
\item[•] a legutolsó N mérkőzés: sikeres első szervák aránya játékosonként - összes (2 érték)
\item[•] a legutolsó N mérkőzés: első szervák győzelmi aránya játékosonként - összes (2 érték)
\item[•] a legutolsó N mérkőzés: break labdák kihasználási aránya játékosonként - összes (2 érték)
\item[•] a legutolsó N mérkőzés: break labdák hárítási aránya játékosonként - összes (2 érték)
\item[•] a legutolsó N mérkőzés: második szervák győzelmi aránya (adogatóként) játékosonként - összes (2 érték)
\item[•] a legutolsó N mérkőzés: második szervák győzelmi aránya (fogadóként) játékosonként - összes (2 érték)
\end{enumerate}

\paragraph{}
A fenti adatokat és szempontokat felhasználva/figyelembe véve a következő eredményeket kaptuk: 
\begin{center}
\begin{tabular}{ |p{4cm}|p{4cm}|  }
 \hline
 \multicolumn{2}{|c|}{A training és a teszt eredménye százalékban - 5000 epoch} \\
 \hline
  Training & Teszt\\
 \hline
 69 & 65  \\
 \hline
\end{tabular}
\end{center}

%%%%%%%%%%%%%%%%%%%%%%%%%%%%%%%%%%%%%%%%%%%%%%%%%%%%%%%%%%%%%%%%%%%%%%%
\section{Harmadik mérföldkő/méréspont}\label{sec:MAT:bev}
\paragraph{}
Ebben a fázisban 66 input adatot használtunk mérkőzésenként, az előzőhöz képest annyi változtatással, hogy a 2. mérföldkőnél bevezetett szempontokat alkalmaztuk a következő esetekben:
\begin{enumerate}
\item[•] a legutolsó M mérkőzésre - az adott talajon (16 érték)
\item[•] a legutolsó P mérkőzésre - az adott tornán  (16 érték)

\end{enumerate}

\paragraph{}
A fenti adatokat és szempontokat felhasználva/figyelembe véve a következő eredményeket kaptuk: 
\begin{center}
\begin{tabular}{ |p{4cm}|p{4cm}|  }
 \hline
 \multicolumn{2}{|c|}{A training és a teszt eredménye százalékban - 5000 epoch} \\
 \hline
  Training & Teszt\\
 \hline
 88 &  85 \\
 \hline
\end{tabular}
\end{center}

%%%%%%%%%%%%%%%%%%%%%%%%%%%%%%%%%%%%%%%%%%%%%%%%%%%%%%%%%%%%%%%%%%%%%%%
\section{Utolsó mérföldkő/méréspont}\label{sec:MAT:bev}
\paragraph{}
Az utolsó fázisban 90 input adatot használtunk mérkőzésenként, az előzőhöz képest a következők hozzáadásával:
\begin{enumerate}
\item[•] a második mérföldkőben bevezetett szempontok - az aktuális tornán az adott meccsig - játékosonként (16 érték)
\item[•] a torna adott szakaszában - győzelmi arány - játékosonként - összes (2 érték)
\item[•] a torna adott szakaszában - győzelmi arány - játékosonként - az adott talajon (2 érték)
\item[•] a torna adott szakaszában - győzelmi arány - játékosonként - az adott tornán (2 érték)
\item[•] a pályán töltött percek aránya játékosonként - az adott tornán (2 érték)

\end{enumerate}

\paragraph{}
A fenti adatokat és szempontokat felhasználva/figyelembe véve a következő eredményeket kaptuk: 
\begin{center}
\begin{tabular}{ |p{4cm}|p{4cm}|  }
 \hline
 \multicolumn{2}{|c|}{A training és a teszt eredménye százalékban - 5000 epoch} \\
 \hline
  Training & Teszt\\
 \hline
 97 & 91  \\
 \hline
\end{tabular}
\end{center}

%%%%%%%%%%%%%%%%%%%%%%%%%%%%%%%%%%%%%%%%%%%%%%%%%%%%%%%%%%%%%%%%%%%%%%%
\section{Összegzés}\label{sec:MAT:bev}
\paragraph{}
Az eredményeket megvizsgálva láthatjuk, hogy nem minden újonnal hozzáadott szempont járt lényeges hatékonyságnövekedéssel, viszont kis mértékben majdnem mindegyikük hozzájárul a végső célunkhoz, amely nem más, mint egy jó tipp megalkotása. Az alkalmazás összesen 27 év mérkőzéseivel dolgozik, amely nagyjából 98.000 mérkőzést jelent, mérkőzésenként 30-35 mutatóval, amelyekből mi magunk is hoztunk létre még újakat. Ezeknek a meccseknek nagyjából a 30-35 százalékát használja fel a program, mint training adat, a tesztelés pedig a 2017-ben lejátszott meccsekkel történik, amely 3800 mérkőzést jelent (természetesen ezek a meccsek nem részei a training-nek).
