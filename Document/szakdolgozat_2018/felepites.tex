%!TEX root = dolgozat.tex
%%%%%%%%%%%%%%%%%%%%%%%%%%%%%%%%%%%%%%%%%%%%%%%%%%%%%%%%%%%%%%%%%%%%%%%
\chapter{Az alkalmazás felépítése}\label{ch:felepites}

\begin{osszefoglal}
	Mint ahogyan az előző fejezetekből is megtudhattuk, az alkalmazás lényege, hogy egy, az általunk kiválasztott mérkőzésről, a kiválasztott játékosok formájától és addigi statisztikáiktól függően egy jóslatot tárjon elénk, amelyben a játékosok százalékos győzelmi esélyeit jeleníti meg. 
	\paragraph{}
	Ez az alkalmazás két fő részből áll, amelyek közül az egyik szintén tovább bontható:
	\begin{enumerate}
		\item[•] Kliens oldali rész
		\item[•] Szerver oldali rész
		\begin{enumerate}
			\item[•] Spring: adatbázis műveletek
			\item[•] Python: neurális háló
		\end{enumerate}
	\end{enumerate}
\end{osszefoglal}

\section{A kliens oldal}
\paragraph{}
A kliens oldal a  React keretrendszer által lett megvalósítva. Az alkalmazás lehetőséget nyújt a felhasználók számára, hogy a tennis 4 nagy tornájának(Grand slams) a nyerteseit, mérkőzéseit megtekintjük táblázatos formában. Ugyanakkor lehetőségünk van tetszés szerint a tornák neve alapján is rákeresni bizonyos eseményekre, tornákra. 
\paragraph{}
Ugyanúgy mint a tornák esetén, a játékosok esetében is lehetőségünk van a név szerinti keresésre, ahol a kiválasztott játékos karrierjének összes fontosabb mozzanatáról találhatunk néhány érdekesebb statisztikai adatot.
\paragraph{}
Az alkalmazás legérdekesebb és egyik legfontosabb része, a játékosok párba/szembeállítása, amikor az egymással szembeni, egymáshoz viszonyított statisztikai adatokat tárja elénk az alkamazás. Ennél a résznél van lehetőségünk a párharc kimenetelének a "megjósoltatására" is.

\subsection{CSS keretrendszer - Bootstrap}
\paragraph{}
Az alkalmazás reszponzív, azaz egy olyan alkalmazás, amely törekszik arra, hogy optimális megjelenítést biztosítson a könnyű olvashatóság és egyszerű navigációval egyidőben. Ezt különböző eszközökön egyaránt próbálja fenntartani az asztali számítógépek képernyőjétől a mobiltelefonok kijelzőjéig egyaránt. Azaz a webalkalmazás alkalmazkodik az őt használó eszköz méreteihez:
\begin{enumerate}
\item[•] A rugalmas felosztású koncepció alapján a honlap minden elemének mérete százalékosan, relatívan van meghatározva.
\item[•] A flexibilis képek úgyszintén a befoglaló elemhez képest, százalékosan határozódnak meg.
\item[•] A media query alkalmazásával megvalósíthatjuk, hogy a weboldalon mindig olyan CSS szabályok lépjenek érvénybe, amelyek a megjelenítő eszközön optimálisak.
\end{enumerate}

\paragraph{}
A Bootstrap egy olyan eszközkészletet kínál, amely előre megírt, multifunkcionálisan alkalmazható, aminek a segítségével átláthatóbban, gyorsabban és hatékonyabban dolgozhatunk. A CSS tulajdonságok és a HTML struktúra mellett számos JavaScript bővítménnyel is rendelkezik, melyek igen rugalmasak!

\begin{lstlisting}[caption=Responsive webtervezés - Bootstrap keretrendszer]
<ul className="nav nav-pills mb-3" id="pills-tab" role="tablist">
  <li className="nav-item">
    <a className="nav-link active font-weight-bold" id="pills-latest-tab" data-toggle="pill" href="#pills-latest" role="tab" aria-controls="pills-latest" aria-selected="true"> 
      Latest 
    </a>
  </li>
  <li className="nav-item">
    <a className="nav-link font-weight-bold" id="pills-grand-slam-tab" data-toggle="pill" href="#pills-grand-slam" role="tab" aria-controls="pills-grand-slam" aria-selected="false"> 
      Grand Slams 
    </a>
  </li>
</ul>
\end{lstlisting}

\paragraph{} A kliensoldal felépítése során fontos szempontnak számított az újrafelhasználható komponensek használata, ezért az alkamazásunkban több helyen is ugyanazokat a komponenseket használtuk, ezáltal is törekedve a kód méretének minimalizálására.

\section{Szerver oldal}
\subsection{Java - Spring}